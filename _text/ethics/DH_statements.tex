\documentclass{article}
\usepackage{graphicx} % Required for inserting images
\usepackage{enumitem}
\usepackage{fancyhdr} % Required for custom headers and footers

% Define the page style with fancyhdr
\pagestyle{fancy}
\fancyhf{} % clear all header and footer fields
\fancyhead[L]{\small Introduction to Digital Humanities} % Small header text on the left
\fancyhead[C]{} % You can add text in the center if needed
\fancyhead[R]{23 April 2024} % You can add text on the right if needed
\fancyfoot[C]{\thepage} % Centered page number in the footer

\title{Digital Humanities \\ \large Some statements, open for debate}
\author{} % You can put your name or leave it empty
\date{} % You can put a date or leave it empty to not display it

\begin{document}

\maketitle

\section*{Responsabilities in Digital Humanities}

\begin{itemize}

    \item Openness about the algorithms used in data processing is crucial for ensuring they do not perpetuate existing societal biases.
    
    \item The inherent biases in historical data collection are too pervasive to allow for truly objective modern data applications.

    \item Bias is data too! Biases aren't bugs, they're features. Data reflects our choices—change the choices, change the data.

    \item Effective data science must account for intersectionality; overlooking this is not just a methodological flaw, but an ethical failure.
        
    \item Corporations that collect and analyze vast amounts of user data are obligated to use this data to benefit society, not just their shareholders.
    
    \item Researchers must consider not only how they use data ethically but also how their data might be used or misused by others after publication.
    
    \item The monetization of personal data by corporations should be more heavily regulated by governments to protect individual rights.
    
    \item Digital Humanists have a responsibility to use their skills for social activism, challenging injustices and promoting equity.

\end{itemize}

\section*{(Digital) Humanities}

\begin{itemize}

    \item All digital humanistic work should be openly accessible, to democratize knowledge beyond the traditional academic gates.

    \item Digital tools should extend the capabilities of humanistic study, not merely serve as high-tech replacements for traditional methods.

    \item Digital methodologies have the potential to fundamentally reshape how we conduct humanities research.

    \item Digital methodologies \textit{should} fundamentally reshape how we conduct humanities research.

    \item Humanities scholars should embrace `maker' cultures that incorporate coding, design, and multimedia artistry into their research methodologies.

    \item Digital Humanists are ``so nice'' because they are not really concerned with theory (more so with methodology). Because of that, there are fewer ``fights'' in the field.
        
    \item Design and architecture should be central to how humanities research questions are formulated and communicated.
    
    \item Digital tools and platforms should fundamentally reshape not just how we conduct humanities research, but also how we communicate and teach these subjects.
    
\end{itemize}

\section*{Rethinking the Humanities}

\begin{itemize}

\item Curation should be valued as much as traditional scholarship in the humanities.

\item Curation should be recognized as a fundamental scholarly practice within humanities, equivalent to traditional narrative scholarship.

\item The future of humanities lies in breaking down the walls between disciplines rather than in further specialization.

\item Text is tyranny! Prioritizing text marginalizes other narratives. Embrace multimedia as the new canon of critical scholarship.

\item Gatekept knowledge is the enemy of progress.

\end{itemize}

\end{document}
